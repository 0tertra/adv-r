\chapter{Releasing a package}

\section{Checking}

\begin{itemize}
\item
  from within R, run \texttt{roxygenise()}, or
  \texttt{devtools::document()} to update documentation
\item
  from the command line, run \texttt{R CMD check}
\end{itemize}

Passing \texttt{R CMD check} is the most frustrating part of package
development, and it usually takes some time the first time. Hopefully by
following the tips elsewhere in this document you'll be in a good place
to start -- in particular, using roxygen and only exporting the minimal
number of functions is likely to save a lot of work.

One place that it is frustrating to have problems with is the examples.
If you discover a mistake, you need to fix it in the roxygen comments,
rerun roxygen and then rerun \texttt{R CMD check}. The examples are one
of the last things checked, so this process can be very time consuming,
particularly if you have more than one bug. The \texttt{devtools}
package contains a function, \texttt{run\_examples} designed to make
this somewhat less painful: all it does is run functions. It also has an
optional parameter which tells it which function to start at - that way
once you've discovered an error, you can rerun from just that file, not
all the files that lead up to.

\section{Version numbers}

The version number of your package increases with subsequent releases of
a package, but it's more than just an incrementing counter -- the way
the number changes with each release can convey information about what
kind of changes are in the package.

An R package version can consist of a series of numbers, each separated
with ``.'' or ``-''. For example, a package might have a version 1.9.
This version number is considered by R to be the same as 1.9.0, less
than version 1.9.2, and all of these are less than version 1.10 (which
is version ``one point ten'', not ``one point one zero). R uses version
numbers to determine whether package dependencies are satisfied. A
package might, for example, import package
\texttt{devtools (\textgreater{}= 1.9.2)}, in which case version 1.9 or
1.9.0 wouldn't work.

The version numbering advice here is inspired in part by
\href{http://semver.org}{Semantic Versionsing} and by the
\href{http://www.x.org/releases/X11R7.7/doc/xorg-docs/Versions.html}{X.Org}
versioning schemes.

A version number consists of up to three numbers,
\emph{}.\emph{}.\emph{}. For version number 1.9.2, 1 is the major
number, 9 is the minor number, and 2 is the patch number. If your
version number is 2.0, then implicit patch number is 0.

As your package evolves, the way the version number changes can reflect
the type of changes in the code:

\begin{itemize}
\itemsep1pt\parskip0pt\parsep0pt
\item
  The major number changes when there are incompatible API changes.
\item
  The minor number changes when there are backward-compatible API
  changes.
\item
  The patch number changes with backwards-compatible fixes.
\item
  Additionally, during development between released versions, the
  package has a sub-patch version number of 99, as in 1.9.0.99. This
  makes it clear for users that they're using a development version of
  the package, as opposed to a formally released version.
\end{itemize}

Remember that these are guidelines. In practice, you might make changes
that fall between the cracks. For example, if you make an
API-incompatible change to a rarely-used part of your code, it may not
deserve a major number change.

\section{Publishing on CRAN}

Once you have passed the checking process, you need to upload your
package to CRAN. The checks will be run again with the latest
development version of R, and on all platforms that R supports - this
means that you should be prepare for more bugs to crop up. Don't get
excited too soon!

\begin{itemize}
\item
  update \texttt{NEWS}, checking that dates are correct. Use
  \texttt{devtools::show\_news} to check that it's in the correct
  format.
\item
  \texttt{R CMD build} then upload to CRAN:
  \texttt{ftp -u ftp://cran.R-project.org/incoming/ package\_name.tar.gz}
\item
  send an email to \texttt{cran@r-project.org}, using the email address
  listed in the DESCRIPTION file. An example email would be something
  like: Hello, I just uploaded package name to CRAN. Please let me know
  if anything goes wrong. Thank you, Me. The subject line should be
  \texttt{CRAN submission PACKAGE VERSION}, this helps the CRAN
  maintainers keep track of the different submissions.
\end{itemize}

Once all the checks have passed you'll get a friendly email from the
CRAN maintainer and you'll be ready to start publicising your package.

\section{Publicising}

Once you've received confirmation that all checks have passed on all
platforms, you have a couple of technical operations to do:

\begin{itemize}
\item
  \texttt{git tag}, so you can mark exactly what version of the code
  this release corresponds to
\item
  bump version in \texttt{DESCRIPTION} and \texttt{NEWS} files
\end{itemize}

Then you need to publicise your package. This is vitally important - for
your hard work to be useful to someone, they need to know that it
exists!

\begin{itemize}
\item
  send release announcement to \texttt{r-packages@stat.math.ethz.ch}. A
  release announcement should consist of a general introduction to your
  package (i.e. why should people care that you released a new version),
  and as well as what's new. I usually make these announcements by
  pasting together the package \texttt{README} and the appropriate
  section from the \texttt{NEWS}.
\item
  announce on twitter, blog etc.
\item
  Finally, don't forget to update your package webpage. If you don't
  have a package webpage -- create one! There you can announce new
  versions, point to help resources, videos and talks about the package.
  If you're using github, I'd recommend using
  \href{http://pages.github.com/}{github pages} to create the website.
\end{itemize}
