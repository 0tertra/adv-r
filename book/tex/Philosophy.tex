\chapter{Package development philosophy}

This book espouses a particular philosophy of package development - it
is not shared by all R developers, but it is one connected to a specific
set of tools that makes package development as easy as possible.

There are three packages we will use extensively:

\begin{itemize}
\item
  \texttt{devtools}, which provides a set of R functions that makes
  package development as easy as possible.
\item
  \texttt{roxygen2}, which translates source code comments into R's
  official documentation format
\item
  \texttt{testthat}, which provides a friendly unit testing framework
  for R.
\end{itemize}

Other styles of package development don't use these packages, but in my
experience they provide a useful trade off between speed and rigour.
That's a theme that we'll see a lot in this chapter: base R provides
rigorous tools that guarantee correctness, but tend to be slow.
Sometimes you want to be able to iterate more rapidly and the tools we
discuss will allow you to do so.

A package doesn't need to be complicated. You can start with a minimal
subset of useful features and slowly build up over time. While there are
strict requirements if you want to publish a package to the world on
CRAN (and many of those requirements are useful even for your own
packages), most packages won't end up on CRAN. Packages are really easy
to create and use once you have the right set of tools.

Anytime you create some reusable set of functions you should put it in a
package. It's the easiest path because packages come with conventions:
you don't need to figure them out for yourself. You'll start with just
your R code in the \texttt{R/} directory, and over time you can flesh it
out with documentation (in \texttt{man/}), compiled code (in
\texttt{src/}), data sets (in \texttt{data/}), and tests (in
\texttt{inst/tests}).

\section{Getting started}

To get started, make sure you have the latest version of R: if you want
to submit your work to CRAN, you'll need to make sure you're running all
checks with the latest R.

You can install the packages you need for this chapter with:

\begin{verbatim}
install.packages("devtools", dependencies = TRUE)
\end{verbatim}

You'll also need to make sure you have the appropriate development tools
installed:

\begin{itemize}
\item
  On Windows, download and install Rtools:
  \url{http://cran.r-project.org/bin/windows/Rtools/}. This is not an R
  package.
\item
  On Mac, make sure you have either XCode (free, available in the app
  store) or the ``Command Line Tools for Xcode'' (needs a free apple id,
  available from \url{http://developer.apple.com/downloads})
\item
  On Linux, make sure you've installed not only R, but the R development
  devtools. This is a Linux package called something like
  \texttt{r-base-dev}.
\end{itemize}

You can check you have everything installed and working by running this
code:

\begin{verbatim}
library(devtools)
has_devel()
\end{verbatim}

It will print out some compilation code (this is needed to help diagnose
problems), but you're only interested in whether it returns
\texttt{TRUE} (everything's ok) or an error (which you need to
investigate further).

\section{Introduction to devtools}

The goal of the devtools package is to make package development as
painless as possible by encoding package building best practices in
functions (so you don't have to remember or even know about them), and
by minimising the iteration time when you're developing a package.

Most of the devtools functions we will use take a path to the package as
their first argument. If the path is omitted, devtools will look in the
current working directory - so for that reason, it's good practice to
have your working directory set to the package directory.

The functions that you'll use most often are those that facilitate the
{[}{[}package development cycle \textbar{}development{]}{]}:

\begin{itemize}
\item
  \texttt{load\_all()}: simulates package installation and loading by
  \texttt{source()}ing all files in the \texttt{R/} directory, compiling
  and linking C, C++ and Fortran files in the \texttt{src/} and
  \texttt{load()}ing data files in the \texttt{data/} directory. More on
  that in {[}{[}package development \textbar{}development{]}{]}
\item
  \texttt{document()}: extracts documentation from source code comments
  and creates \texttt{Rd} files in the \texttt{man/} directory. You can
  use \texttt{dev\_help()} and \texttt{dev\_example()} instead of
  \texttt{help()} and \texttt{example()} to preview these files without
  installing the package. More on that in {[}{[}documentation{]}{]}.
\item
  \texttt{test()}: runs all unit tests in the \texttt{inst/tests/}
  directory and reports the results. More on that in
  {[}{[}testing{]}{]}.
\end{itemize}

Other functions mimic standard R commands that you run from the command
line:

\begin{itemize}
\item
  \texttt{build()} is equivalent to \texttt{R CMD build} and bundles
  package. See {[}{[}package-basics{]}{]} for more about what a bundled
  package is.
\item
  \texttt{install()} is equivalent to \texttt{R CMD INSTALL} and
  installs a package into a local library. Learn more about the
  installation process in {[}{[}package-basics{]}{]}.
\item
  \texttt{check()} is equivalent to \texttt{R CMD check}, and runs a
  large set of automated tests against your package. Read more about
  checking as part of the {[}{[}release{]}{]} process.
\end{itemize}

These tools should be more reliable than running the equivalent commands
in the terminal (and much easier to use if you're not familiar with the
terminal). They do more to ensure that command-line R is running in
exactly the same way as your R GUI. They check that you're running the
same version of R, with the same library paths, and with a standard
collation order. These are things you don't need to worry about most of
the time, but if they ever trip you up, it can take hours to figure out
the source of the problem. \texttt{check()} and \texttt{install()} also
run build first, which is recommended best practice.

There are two other functions that you use less commonly, only at the
start and the end of package development:

\begin{itemize}
\item
  \texttt{create()} creates a new package, and fills it out with the
  basics (this is the \texttt{devtools} equivalent of
  \texttt{package.skeleton})
\item
  \texttt{release()} checks the package, checks that you've done what
  the CRAN maintainers expect, uploads to CRAN and drafts an email for
  the CRAN maintainers
\end{itemize}
