\chapter{Git and github}

\section{Git}

Using a source code control system, like git, is highly recommended
because it makes it easy to:

\begin{itemize}
\item
  incorporate contributions from multiple developers working on the code
  at the same time
\item
  rewind time to undo mistakes or see what has changed between working
  code and broken code
\end{itemize}

In this document, I'll describe the use of git and github because they
are the tools that I am most familiar with. There are many others (like
subversion, mercurial and bazaar) that offer similar capabilities - the
choice of git is somewhat arbitrary but the skills will readily transfer
to other systems.

I'll just give you the basics, and give pointers to places where you can
learn more advanced techniques.

\url{http://tbaggery.com/2008/04/19/a-note-about-git-commit-messages.html}

\subsection{The basics}

\begin{itemize}
\itemsep1pt\parskip0pt\parsep0pt
\item
  \texttt{git init}
\item
  \texttt{git add}
\item
  \texttt{git commit}
\item
  \texttt{git push}
\item
  \texttt{git pull -\/-rebase}
\end{itemize}

\section{Github}

\begin{itemize}
\itemsep1pt\parskip0pt\parsep0pt
\item
  git enabled wiki (which these documents have been written in)
\item
  makes it easy for others to send in patches
\item
  ticketing system for tracking bugs
\item
  RSS feed of changes, and line-by-line comments are very useful for
  working with collaborators
\end{itemize}

Additionally github gives you free hosting for public repositories. You
only have to pay if you want private repositories with private
collaborators.
